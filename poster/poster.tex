\documentclass{beamer}

\usepackage[orientation=portrait,size=a1,scale=1.5]{beamerposter}
\setbeamercolor{block title}{fg=white,bg=title.fg!75!white}
\setbeamercolor{block body}{bg=title.fg!3!white}
\addtobeamertemplate{block end}{}{\vspace*{2ex}} % White space under blocks
% https://tex.stackexchange.com/a/160328
\addtobeamertemplate{headline}{}{
	\begin{tikzpicture}[remember picture, overlay]
		\node [anchor=north west, inner sep=2em, yshift=-6em] at (current page.north west) {\usebeamertemplate*{logo left}};
		\node [anchor=north east, inner sep=2em, yshift=-6em] at (current page.north east) {\usebeamertemplate*{logo right}};
	\end{tikzpicture}
}
\addtobeamertemplate{title page}{}{\vspace{-2em}}
\setbeamertemplate{navigation symbols}{}

\setbeamerfont{title}{size=\huge,series=\bfseries}
\setbeamerfont{subtitle}{size=\large,series=\mdseries}
\setbeamerfont{author}{size=\Large,series=\bfseries}
\setbeamerfont{institute}{size=\large}

\usepackage[utf8]{inputenc}
\usepackage[english]{babel}
\usepackage{csquotes}

\addtobeamertemplate{block begin}{}{\setlength{\parskip}{10pt}\vspace{-\parskip}} % hack

\usepackage[backend=biber,sorting=none]{biblatex}
\ExecuteBibliographyOptions{giveninits=true,maxnames=4}
\ExecuteBibliographyOptions{doi=false,isbn=false,eprint=false,url=false}
%\DeclareFieldFormat{url}{\footnotesize\url{#1}} % no URL: prefix
\DeclareFieldFormat{url}{} % no url field, url=false doesn't work in beamer...
\DeclareFieldFormat{urldate}{\mkbibparens{#1}} % no urldate prefix word
\renewbibmacro{in:}{}
\addbibresource{poster.bib}

\setbeamercolor{bibliography entry author}{fg=black}
\setbeamercolor{bibliography entry note}{fg=black}
\setbeamerfont{bibliography item}{size=\small} % unneeded?
\renewcommand*{\bibfont}{\small}

% https://tex.stackexchange.com/a/148696
\usepackage{ragged2e}
\addtobeamertemplate{block begin}{}{\justifying}

\usepackage{graphicx}
\usepackage{subcaption}

\usepackage{tikz}
\usetikzlibrary{positioning,arrows.meta}

\usepackage[outputdir=build]{minted}
\setminted{
	autogobble=true
}

\newminted[smtlib]{lisp}{fontsize=\small}


\title{Satisfiability Modulo Integer Difference Logic}
\subtitle{MTAT.03.238, Advanced Algorithmics}
\author{Simmo Saan}
\institute{Institute of Computer Science, University of Tartu}
\date{}

\setbeamertemplate{logo left}{\includegraphics[height=15em]{tartu_ylikool-logo_ja_v2rvid-cmyk-07_copy.png}}
\setbeamertemplate{logo right}{\includegraphics[height=12.5em]{ita_small-logo-eng.png}}

\begin{document}

%\nocite{*}

\begin{frame}[fragile,t]
\maketitle

\begin{columns}[t,onlytextwidth]
	\begin{column}{0.485\textwidth}
		\begin{block}{Introduction}
			Boolean satisfiability (SAT) problems, which consist of purely boolean algebra expressions, can be solved by SAT solvers. While many complicated problems can be described as SAT problems, describing the problem purely with booleans and constructing the respective conjunctive normal form (CNF) can be difficult and inefficient.

			Being able to use integer variables and conditions makes modeling of many problems much easier and more intuitive. \textbf{Satisfiability modulo theories (SMT)} is a generalization of SAT, where the logical expressions may, besides boolean variables, contain, for example, integer variables and conditions between them. Such problems are solved by SMT solvers.

			The goal of the project was to understand is to understand how SMT solvers work and implement a simple solver.
		\end{block}
	
		\begin{block}{DPLL algorithm~\cite{DPLLmethod}}
			Davis–Putnam–Logemann–Loveland (DPLL) algorithm is a \emph{complete} SAT/SMT solving algorithm based on backtracking search, where call consists of three steps:
			\begin{enumerate}
				\item \textbf{Conflict checking:} if there is a conflict, backtrack.
				\item \textbf{Unit propagation:} if a clause contains only one literal, then assume it to be true.
				\item \textbf{Splitting:} choose a variable and split into two recursive searches, in the first one assume the variable to be true, in the second one false.
			\end{enumerate}
		
			DPLL allows for significant improvements like conflict-driven clause learning (CDCL), which is used by state of the art SAT/SMT solvers.
		\end{block}
	
		\begin{block}{Integer difference logic~\cite{slides}}
			Integer difference logic (IDL) uses integer variables and constraints in the form $x - y \leq n$, where $x, y$ are integer variables and $n$ an integer constant. Systems of such difference constraints can, for example, encode various scheduling problems.
			
			Furthermore, many other integer constraints can be converted to IDL, for example:
			\begin{itemize}
				\item $x - y < n \quad\to\quad x - y \leq n - 1$,
				\item $x - y \geq n \quad\to\quad y - x \leq -n$,
				\item $x - y = n \quad\to\quad (x - y \leq n) \land (x - y \geq n)$,
				\item $x - y \neq n \quad\to\quad (x - y < n) \lor (x - y > n)$,
				\item $x \bowtie y + n \quad\to\quad x - y \bowtie n$,
				\item $\quad\;\;\: x \bowtie n \quad\to\quad x - \mathbf{Z} \bowtie n$,
			\end{itemize}
			where ${\bowtie} \in \{\leq, <, \geq, >, =, \neq\}$ and $\mathbf{Z}$ is an artifical zero variable.
		\end{block}

		\begin{block}{Bellman-Ford algorithm~\cite{CLRS}}
			Systems of difference constraints can be transformed into constraint graphs with an extra vertex. The Bellman-Ford single-source shortest paths algorithms can be used to find a solution from the distances, or detect a conflict from negative weight cycle.

			\begin{figure}[h]
				\centering
				\begin{subfigure}{0.3\linewidth}
					\centering
					\begin{align*}
						x_1 - x_2 &\leq 2 \\
						x_2 - x_3 &\leq 1 \\
						x_3 - x_1 &\leq -1 \\
					\end{align*}
				\end{subfigure}
				~
				\begin{subfigure}{0.5\linewidth}
					\centering
					\begin{tikzpicture}[
						node distance=5ex and 10ex,
						line width=0.1ex,
						on grid,
						vertex/.style={
							draw,circle,inner sep=0.5ex
						},
						arrows={-Stealth[scale=2]},auto
					]
						\node[vertex] (x0) {$x_0$};
						\node[vertex] (x1) [right=of x0] {$x_1$};
						\node[vertex] (x2) [above right=of x1] {$x_2$};
						\node[vertex] (x3) [below right=of x1] {$x_3$};

						\path
							(x0) edge node {0} (x1)
							     edge [bend left] node [swap] {0} (x2)
							     edge [bend right] node [near start] {0} (x3)
							(x1) edge node [swap] {-1} (x3)
							(x2) edge node {2} (x1)
							(x3) edge node [swap] {1} (x2);
					\end{tikzpicture}
				\end{subfigure}
			\end{figure}

			Notable optimizations are
			\begin{itemize}
				\item Avoiding the use of extra vertex by initializing all distances to 0.
				\item Returning early if no edge is relaxed.
			\end{itemize}
		\end{block}
	\end{column}

	\begin{column}{0.485\textwidth}
		\begin{block}{SMT solvers~\cite{slides}}
			The modern standard architecture of SMT solvers is the following:
			\begin{itemize}
				\item A \textbf{SAT solver} (e.g. DPLL algorithm) handles solving of the boolean expressions, without having to deal with the logic constraints within.
				\item A \textbf{logic solver} (e.g. Bellman-Ford algorithm for IDL) handles solving of the logic itself, without having to deal with boolean satisfiability.
			\end{itemize}
			The two independent solving procedures are combined by having the SAT solver call the logic solver when it needs to, for example, to check for check for conflicts.
		\end{block}
	
		\begin{block}{Dinesman's multiple-dwelling problem} %~\cite{SICP}}
			Problem definition in English and SMT-LIB language:
			\begin{enumerate}
				\setcounter{enumi}{-1} % start from 0
				
				\item Baker, Cooper, Fletcher, Miller, and Smith live on different floors of an apartment house that contains only five floors.
				\begin{smtlib}
					(assert (distinct baker cooper fletcher miller smith))
					(assert (<= 1 baker 5))
					(assert (<= 1 cooper 5))
					(assert (<= 1 fletcher 5))
					(assert (<= 1 miller 5))
					(assert (<= 1 smith 5))
				\end{smtlib}
				
				\item Baker does not live on the top floor.
				\begin{smtlib}
					(assert (distinct baker 5))
				\end{smtlib}
			
				\item Cooper does not live on the bottom floor.
				\begin{smtlib}
					(assert (distinct cooper 1))
				\end{smtlib}
				
				\item Fletcher does not live on either the top or the bottom floor.
				\begin{smtlib}
					(assert (and (distinct fletcher 1)
					             (distinct fletcher 5)))
				\end{smtlib}
			
				\item Miller lives on a higher floor than does Cooper.
				\begin{smtlib}
					(assert (> miller cooper))
				\end{smtlib}
			
				\item Smith does not live on a floor adjacent to Fletcher's.
				\begin{smtlib}
					(assert (and (distinct smith (+ fletcher 1))
					             (distinct smith (+ fletcher (- 1)))))
				\end{smtlib}
			
				\item Fletcher does not live on a floor adjacent to Cooper's.
				\begin{smtlib}
					(assert (and (distinct fletcher (+ cooper 1))
					             (distinct fletcher (+ cooper (- 1)))))
				\end{smtlib}
			\end{enumerate}
		
			SMT-LIB commands for checking satisfiability and respective variable values, and their outputs from the implemented SMT solver for the given problem:
			\begin{smtlib}
				(check-sat)
				sat
				(get-model)
				((cooper 2) (fletcher 4) (baker 3) (miller 5) (smith 1))
			\end{smtlib}
		\end{block}
	
		\begin{block}{Conclusion}
			The implemented SMT solver:
			\begin{enumerate}
				\item takes problem definition in SMT-LIB language,
				\item converts constraints to difference constraints (if possible),
				\item converts boolean expression to CNF,
				\item uses DPLL to solve the boolean parts of the problem,
				\item uses Bellman-Ford to solve IDL parts of the problem for DPLL.
			\end{enumerate}

			Its correctness has been tested on logic puzzles like Dinesman's multiple-dwelling puzzle, Philips's liars puzzle, $n$-queens puzzle and zebra puzzle (Einstein's puzzle).

			\textbf{Project code repository: \url{https://github.com/sim642/algosmt}}.
		\end{block}
	
		\begin{block}{References}
			\setbeamertemplate{bibliography item}[text]
			\printbibliography[heading=none]
		\end{block}
	\end{column}
\end{columns}

%\vspace{0.5em}
%
%\begin{block}{References}
%	\setbeamertemplate{bibliography item}[text]
%	\printbibliography[heading=none]
%\end{block}

\end{frame}

\end{document}